%-------------------------------------------------------------------------------
%	SECTION TITLE
%-------------------------------------------------------------------------------
\cvsection{Projects \& Achievements}


%-------------------------------------------------------------------------------
%	CONTENT
%-------------------------------------------------------------------------------
\begin{cventries}


% \cventry
%     {Developer}
%     {JavsScript30}
%     {Melbourne, Australia}
%     {Aug 2020 - Feb 2021}
%     {
%     \begin{cvitems}
%     \item \href{https://github.com/kongpeter/JavaScript30}{Github Link: JS30}
%     \item Build 30 JavaScript projects without any frameworks in 30 days.
%     \end{cvitems}
%     }

\cventry
    {Core Developer} % Affiliation/role
    {Component Library Development} % Organization/group
    {TikTok, Shanghai} % Location
    {April 2022 – June 2023} % Date(s)
    {
      \begin{cvitems} % Description(s) of experience/contributions/knowledge
        \item Created 30+ reusable UI components with UI designers by using React and JavaScript, resulting in a 30\% reduction in development time.
        \item Provided technical on-call support for developers, resolving over 80 issues.
        \item Enhanced Daily Component Library Iterations with CI/CD Integration, elevating iteration efficiency by 30\%.
        \item Developed a Yarn3 based plugin for Component Library release workflow: automated npm package release, version updates, changelog generation, pre-release and release decisions, and Git commit tracking.
        \item Implemented the plugin for component library CI publishing, resulting in controlled permissions and process standardization.
      \end{cvitems}
    }


\cventry
    {Core Developer}
    {Refactored TikTok E-Commerce Real-time Chat Application}
    {TikTok, Shanghai}
    {Fed 2022 - June 2022}
    {
        \begin{cvitems}
        % \item \href{https://github.com/kongpeter/Child-of-Now}{Github Link: Child of now}
        \item Worked with product managers, backend team, QA testers, refactored TikTok E-Commerce Real-time Chat Application using new cross-platform framework Lynx to overcome performance issues of Webview in mobile apps.
        \item Implemented optimization techniques, including dual-thread rendering, offline caching, and code caching, leading to a significant improvement in overall page performance. Android LCP (Largest Contentful Paint) decreased from 2.5 seconds to 1.2 seconds, and iOS LCP decreased from 1.5 seconds to 0.5 seconds.
        \item The project resulted in significant business benefits, including a better performance with average LCP 0.9s, reduced average user stay time (from 15s -> 11s), and improved communication efficiency between consumers and customer service (user review rate 83\% -> 85\%).
        \end{cvitems}
    }


% \cventry
%     {Owner} % Affiliation/role
%     {IM Performance Optimization with Production Quality Building} % Organization/group
%     {Shanghai} % Location
%     {June 2022 – Dec 2022} % Date(s)
%     {
%       \begin{cvitems} % Description(s) of experience/contributions/knowledge
%         \item In response to stability-related issues in IM, a comprehensive review was conducted, which included building core link monitoring, PV real-time alarming, OOM rate, and JSB monitoring. With the assistance of these new monitoring, production issues were accurately found, the identification rate for production issue increased from 50\%(4/8) to 90\%(12/13).
%         \item Developed internal component library Chat-Design while improving performance and reducing npm package size. Resultant optimizations led to a 36\% reduction in package size (146kb), and decreased LCP about 200ms in both Android and IOS.
%       \end{cvitems}
%     }

\cventry
    {Owner} % Affiliation/role
    {React Performance Optimization} % Organization/group
    {TikTok, Shanghai} % Location
    {Dec 2021 – Mar 2022} % Date(s)
    {
      \begin{cvitems} % Description(s) of experience/contributions/knowledge
        \item Organized and optimized package dependencies, removed unnecessary DI injections, reduced video player package size, implemented an automated solution for skeleton screens, introduced a caching mechanism for historical messages. 
        \item Achieved a substantial 40\% decrease in Android FCP90\% (First Contentful Paint) from 4000ms to 2300ms, IOS FCP90\% from 2100ms to 1700ms.
      \end{cvitems}
    }



\cventry
    {Owner} % Affiliation/role
    {Built IM Platform Capabilities} % Organization/group
    {TikTok, Shanghai} % Location
    {April 2022 – May 2023} % Date(s)
    {
      \begin{cvitems} % Description(s) of experience/contributions/knowledge
        \item Migrated IM features to over 10 apps within the company, such as Toutiao, Xigua video, Huoshan video, Fanqie novel, which provided customers with a complete shopping experience.
        \item Created SOP documentation for Migration reduced 60\% development time.
      \end{cvitems}
    }



%---------------------------------------------------------
% \cventry
%     {Full-Stack Developer} % Affiliation/role
%     {Distributed System Project} % Organization/group
%     {Melbourne, Australia} % Location
%     {July 2019 – Sep 2019} % Date(s)
%     {
%       \begin{cvitems} % Description(s) of experience/contributions/knowledge
%         % \item \href{https://github.com/kongpeter/Multi-Threaded-Dictionary-Server}{Github Link: Multi Threaded Dictionary Server}
%         \item {Designed a multi-threaded server allows concurrent clients to do some basic operations, adding, searching and removing word in existed dictionary file in server.}
%       \end{cvitems}
%     }
%---------------------------------------------------------


% \cventry
%     {Developer} % Affiliation/role
%     {SQL Project} % Organization/group
%     {Melbourne, Australia} % Location
%     {March 2019 – June 2019} % Date(s)
%     {
%       \begin{cvitems} % Description(s) of experience/contributions/knowledge
%         \item {Build a simple Database based on MySQL}
%         \item {Solved some significant query problems based on this Database}
%       \end{cvitems}
%     }


%---------------------------------------------------------
% \cventry
%     {Developer \& Researcher} % Affiliation/role
%     {Image Processing approaches for species identification of pests} % Organization/group
%     {Suzhou, China} % Location
%     {June 2018 – Feb 2019} % Date(s)
%     {
%       \begin{cvitems} % Description(s) of experience/contributions/knowledge
%         \item{Developed an Android application and algorithms to identify the species of pests. The core algorithms were based on deep learning and neural-network.}
%       \end{cvitems}
%     }
    
    %---------------------------------------------------------

%---------------------------------------------------------
%   \cventry
%     {Developer} % Affiliation/role
%     {Image Processing Project} % Organization/group
%     {Suzhou, China} % Location
%     {Feb 2018 – June 2018} % Date(s)
%     {
%       \begin{cvitems} % Description(s) of experience/contributions/knowledge
%         \item \href{https://github.com/kongpeter/Digital-Image-Processing}{Github Link: Image Processing}
%         \item {Implemented machine learning on digital images in Matlab}
%         %\item {Design algorithms to solve image processing problems}
%       \end{cvitems}
%     }



%---------------------------------------------------------
%   \cventry
%     {Researcher \& Core Member \& Developer} % Affiliation/role
%     {Qt Project} % Organization/group
%     {Suzhou, China} % Location
%     {June 2017 – May 2018} % Date(s)
%     {
%       \begin{cvitems} % Description(s) of experience/contributions/knowledge
%         \item {Design an application based on Qt}
%         \item {The application could remotely control semiconductor analyzer B1500A}
%         %\item {Help researchers use this application to remote control B1500A}
%       \end{cvitems}
%     }


% %---------------------------------------------------------
%   \cventry
%     {Researcher \& Programmer} % Affiliation/role
%     {HDL Project} % Organization/group
%     {Suzhou, China} % Location
%     {Sep 2017 - Feb 2018} % Date(s)
%     {
%       \begin{cvitems} % Description(s) of experience/contributions/knowledge
%         \item {Design a single-cycle CPU using verilog}
%         \item {The instruction set is MIPS}
%       \end{cvitems}
%     }



% %---------------------------------------------------------
%   \cventry
%     {Researcher \& Programmer} % Affiliation/role
%     {Optimization Project} % Organization/group
%     {Suzhou, China} % Location
%     {Sep 2017 - Feb 2018} % Date(s)
%     {
%       \begin{cvitems} % Description(s) of experience/contributions/knowledge
%         \item {Optimize mathematical problems based in Matlab}
%         \item {Design an optimized algorithm to solve mathematical problem}
%       \end{cvitems}
%     }





\end{cventries}